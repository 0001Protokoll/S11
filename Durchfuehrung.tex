%%\documentclass[a4paper, 12pt]{scrreprt}

\documentclass[a4paper, 12pt, bibliography=totocnumbered]{scrartcl}
%usepackage[german]{babel}
\usepackage{microtype}
%\usepackage{amsmath}
%usepackage{color}
\usepackage[utf8]{inputenc}
\usepackage[T1]{fontenc}
\usepackage{wrapfig}
\usepackage{lipsum}% Dummy-Text
\usepackage{multicol}
\usepackage{alltt}
%%%%%%%%%%%%bis hierhin alle nötigen userpackage
\usepackage{tabularx}
\usepackage[utf8]{inputenc}
\usepackage{amsmath}
\usepackage{amsfonts}
\usepackage{amssymb}

%\usepackage{wrapfig}
\usepackage[ngerman]{babel}
\usepackage[left=25mm,top=25mm,right=25mm,bottom=25mm]{geometry}
%\usepackage{floatrow}
\setlength{\parindent}{0em}
\usepackage[font=footnotesize,labelfont=bf]{caption}
\numberwithin{figure}{section}
\numberwithin{table}{section}
\usepackage{subcaption}
\usepackage{float}
\usepackage{url}
%\usepackage{fancyhdr}
\usepackage{array}
\usepackage{geometry}
%\usepackage[nottoc,numbib]{tocbibind}
\usepackage[pdfpagelabels=true]{hyperref}
\usepackage[font=footnotesize,labelfont=bf]{caption}
\usepackage[T1]{fontenc}
\usepackage {palatino}
%\usepackage[numbers,super]{natbib}
%\usepackage{textcomp}
\usepackage[version=4]{mhchem}
\usepackage{subcaption}
\captionsetup{format=plain}
\usepackage[nomessages]{fp}
\usepackage{siunitx}
\sisetup{exponent-product = \cdot, output-product = \cdot}
\usepackage{hyperref}
\usepackage{longtable}
\newcolumntype{L}[1]{>{\raggedright\arraybackslash}p{#1}} % linksbündig mit Breitenangabe
\newcolumntype{C}[1]{>{\centering\arraybackslash}p{#1}} % zentriert mit Breitenangabe
\newcolumntype{R}[1]{>{\raggedleft\arraybackslash}p{#1}} % rechtsbündig mit Breitenangabe
\usepackage{booktabs}
\renewcommand*{\doublerulesep}{1ex}
\usepackage{graphicx}
\usepackage{chemformula}
\usepackage{epstopdf}


\usepackage[backend=bibtex, style=chem-angew, backref=none, backrefstyle=all+]{biblatex}
\addbibresource{Literatur.bib}
\defbibheading{head}{\section{Literatur}\label{sec:Lit}} 
\let\cite=\supercite
%\begin{document}

\section{Durchführung}
Es wurden $100$~ml einer Stammlösung von Kristallviolett mit einer Konzentration von $c = 1.00 \cdot 10^{-4}$~M in Ethanol hergestellt. Durch Verdünnen der ersten Stammlösung im Verhältnis $1:10$ wurde die zweite Stammlösung mit einer Konzentration von $c = 1.00 \cdot 10^{-5}$~M hergestellt. Diese Lösung diente als Basis einer Verdünnungsreihe bei der jeweils $10$~ml folgender Konzentrationen verdünnt wurden: $c=1.00 \cdot 10^{-6}$~M, $3.00 \cdot 10^{-6},$~M, $5.00 \cdot 10^{-6}$~M, $8.00 \cdot 10^{-6}$~M, $1.50 \cdot 10^{-5}$~M, $5.00 \cdot 10^{-5}$~M und $8.00 \cdot 10^{-5}$~M. Die einzelnen Verdünnungen wurden dann in dem Spektrophotometer JASCO V-$750$ hinsichtlich der Absorption untersucht. In der der Referenzküvette befand sich der gleiche Ethanol mit dem die Verdünnungsreihe gemacht wurde. Anschießend wurden die Messdaten mittels IGOR Pro ausgewertet.



%\end{document}
