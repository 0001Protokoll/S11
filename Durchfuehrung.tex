%\input{usepackage.tex}
%\begin{document}

\section{Durchführung}
Es wurden $100$~ml einer Stammlösung von Kristallviolett mit einer Konzentration von $c = 1.00 \cdot 10^{-4}$~M in Ethanol hergestellt. Durch Verdünnen der ersten Stammlösung im Verhältnis $1:10$ wurde die zweite Stammlösung mit einer Konzentration von $c = 1.00 \cdot 10^{-5}$~M hergestellt. Diese Lösung diente als Basis einer Verdünnungsreihe bei der jeweils $10$~ml folgender Konzentrationen verdünnt wurden: $c=1.00 \cdot 10^{-6}$~M, $3.00 \cdot 10^{-6},$~M, $5.00 \cdot 10^{-6}$~M, $8.00 \cdot 10^{-6}$~M, $1.50 \cdot 10^{-5}$~M, $5.00 \cdot 10^{-5}$~M und $8.00 \cdot 10^{-5}$~M. Die einzelnen Verdünnungen wurden dann in dem Spektrophotometer JASCO V-$750$ hinsichtlich der Absorption untersucht. In der der Referenzküvette befand sich der gleiche Ethanol mit dem die Verdünnungsreihe gemacht wurde. Anschießend wurden die Messdaten mittels IGOR Pro ausgewertet.



%\end{document}
