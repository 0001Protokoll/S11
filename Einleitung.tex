%\input{usepackage.tex}
%\begin{document}

\section{Einleitung}

Aufgrund der Interaktion von  Materie mit elektromagnetischer Strahlung ist es möglich diese zu Unterschuchen und somit zu charakterisieren. Dies kann auf qualitative und/oder auf quantitative Weise geschehen. Im Falle der UV/Vis-Spektroskopie nutzt man die Absorption vor allem von Molekülen und Komplexen im sichtbaren  sowie im ultravioletten Bereich des Lichtes. Mit dem Lambert-Beerschen Gesetz wird die Beziehung zwischen der Konzentration und der Extinktion beschieben. Dieses Gesetz soll im Folgenden am Beispiel des Kristallviolett untersucht werden.


%\end{document}
